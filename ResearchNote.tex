% Options for packages loaded elsewhere
\PassOptionsToPackage{unicode}{hyperref}
\PassOptionsToPackage{hyphens}{url}
\PassOptionsToPackage{dvipsnames,svgnames,x11names}{xcolor}
%
\documentclass[
  12pt,
]{ctexart}
\title{能力不够压力来凑:数字治理能力下的压力型体制}
\usepackage{etoolbox}
\makeatletter
\providecommand{\subtitle}[1]{% add subtitle to \maketitle
  \apptocmd{\@title}{\par {\large #1 \par}}{}{}
}
\makeatother
\subtitle{------基于Covid19期间的春节返乡政策的混合方法研究}
\author{孙宇飞\footnote{清华大学政治学系博士生,联系电话:18638750921,邮箱:\href{mailto:sunyf20@mails.tsinghua.edu.cn}{\nolinkurl{sunyf20@mails.tsinghua.edu.cn}}}}
\date{}

\usepackage{amsmath,amssymb}
\usepackage{lmodern}
\usepackage{iftex}
\ifPDFTeX
  \usepackage[T1]{fontenc}
  \usepackage[utf8]{inputenc}
  \usepackage{textcomp} % provide euro and other symbols
\else % if luatex or xetex
  \usepackage{unicode-math}
  \defaultfontfeatures{Scale=MatchLowercase}
  \defaultfontfeatures[\rmfamily]{Ligatures=TeX,Scale=1}
\fi
% Use upquote if available, for straight quotes in verbatim environments
\IfFileExists{upquote.sty}{\usepackage{upquote}}{}
\IfFileExists{microtype.sty}{% use microtype if available
  \usepackage[]{microtype}
  \UseMicrotypeSet[protrusion]{basicmath} % disable protrusion for tt fonts
}{}
\makeatletter
\@ifundefined{KOMAClassName}{% if non-KOMA class
  \IfFileExists{parskip.sty}{%
    \usepackage{parskip}
  }{% else
    \setlength{\parindent}{0pt}
    \setlength{\parskip}{6pt plus 2pt minus 1pt}}
}{% if KOMA class
  \KOMAoptions{parskip=half}}
\makeatother
\usepackage{xcolor}
\IfFileExists{xurl.sty}{\usepackage{xurl}}{} % add URL line breaks if available
\IfFileExists{bookmark.sty}{\usepackage{bookmark}}{\usepackage{hyperref}}
\hypersetup{
  pdftitle={能力不够压力来凑:数字治理能力下的压力型体制},
  pdfauthor={孙宇飞},
  colorlinks=true,
  linkcolor={Maroon},
  filecolor={Maroon},
  citecolor={Blue},
  urlcolor={Blue},
  pdfcreator={LaTeX via pandoc}}
\urlstyle{same} % disable monospaced font for URLs
\usepackage[margin=1in]{geometry}
\usepackage{longtable,booktabs,array}
\usepackage{calc} % for calculating minipage widths
% Correct order of tables after \paragraph or \subparagraph
\usepackage{etoolbox}
\makeatletter
\patchcmd\longtable{\par}{\if@noskipsec\mbox{}\fi\par}{}{}
\makeatother
% Allow footnotes in longtable head/foot
\IfFileExists{footnotehyper.sty}{\usepackage{footnotehyper}}{\usepackage{footnote}}
\makesavenoteenv{longtable}
\usepackage{graphicx}
\makeatletter
\def\maxwidth{\ifdim\Gin@nat@width>\linewidth\linewidth\else\Gin@nat@width\fi}
\def\maxheight{\ifdim\Gin@nat@height>\textheight\textheight\else\Gin@nat@height\fi}
\makeatother
% Scale images if necessary, so that they will not overflow the page
% margins by default, and it is still possible to overwrite the defaults
% using explicit options in \includegraphics[width, height, ...]{}
\setkeys{Gin}{width=\maxwidth,height=\maxheight,keepaspectratio}
% Set default figure placement to htbp
\makeatletter
\def\fps@figure{htbp}
\makeatother
\setlength{\emergencystretch}{3em} % prevent overfull lines
\providecommand{\tightlist}{%
  \setlength{\itemsep}{0pt}\setlength{\parskip}{0pt}}
\setcounter{secnumdepth}{5}
\ifLuaTeX
  \usepackage{selnolig}  % disable illegal ligatures
\fi

\begin{document}
\maketitle

\begin{verbatim}
## 
## Call:
## glm(formula = load ~ 治理能力 + `医院、卫生院床位数_全市_张` + 
##     ConfirmedCases_Period_City + ConfirmedCases_Period_Province + 
##     diff202012Covid + peer_pressure + 地区生产总值增长率_全市_百分比 + 
##     年末总人口_全市_万人, family = poisson(link = "log"), 
##     data = fulldata)
## 
## Deviance Residuals: 
##     Min       1Q   Median       3Q      Max  
## -1.7564  -1.0677  -0.8376   0.4084   3.1141  
## 
## Coefficients:
##                                  Estimate Std. Error z value Pr(>|z|)   
## (Intercept)                     1.620e+00  9.668e-01   1.675  0.09388 . 
## 治理能力                       -6.764e-02  2.838e-02  -2.383  0.01716 * 
## `医院、卫生院床位数_全市_张`   -3.975e-05  1.740e-05  -2.285  0.02233 * 
## ConfirmedCases_Period_City      1.757e-03  9.171e-04   1.916  0.05534 . 
## ConfirmedCases_Period_Province  1.161e-03  5.331e-04   2.178  0.02939 * 
## diff202012Covid                -8.585e-03  4.190e-03  -2.049  0.04046 * 
## peer_pressure                   3.175e-02  1.252e-02   2.535  0.01124 * 
## 地区生产总值增长率_全市_百分比 -8.530e-03  8.414e-02  -0.101  0.91925   
## 年末总人口_全市_万人            2.221e-03  8.015e-04   2.771  0.00558 **
## ---
## Signif. codes:  0 '***' 0.001 '**' 0.01 '*' 0.05 '.' 0.1 ' ' 1
## 
## (Dispersion parameter for poisson family taken to be 1)
## 
##     Null deviance: 163.93  on 93  degrees of freedom
## Residual deviance: 132.08  on 85  degrees of freedom
##   (13 observations deleted due to missingness)
## AIC: 225.87
## 
## Number of Fisher Scoring iterations: 6
\end{verbatim}

这是政治表态,所以要把压力体现方式分为几个维度,比如政治表态;具体措施;会议频度;领导人出现情况。然后根据自己掌握的材料,来说明选取了哪几个维度来测量。在这里压力就是``政令的统一和落实''

\begin{itemize}
\tightlist
\item
  Level

  \begin{itemize}
  \tightlist
  \item
    1 = 基本通行政策(测体温、健康码、健康告知书、健康问询、健康提醒)
  \item
    2 = 健康报备
  \item
    3 = 核酸检测/抗体检测
  \item
    4 = 健康监测
  \item
    5 = 居家隔离
  \item
    6 = 集中隔离
  \end{itemize}
\end{itemize}

\hypertarget{ux6587ux7ae0ux5206ux6790ux903bux8f91}{%
\section{文章分析逻辑}\label{ux6587ux7ae0ux5206ux6790ux903bux8f91}}

政策加码过程:中央 - 省 - 地级市 - 基层

政策加码主体:下一级政府

中央 - \textbf{省}:公开政策比较

省 - \textbf{地级市}-政策的层面:公开政策比较

省 - \textbf{地级市}-执行的层面:公开政策与政务留言板的数值差异比较

\hypertarget{ux5206ux6790ux6a21ux578b}{%
\section{分析模型}\label{ux5206ux6790ux6a21ux578b}}

政策加码量(地级市) \textasciitilde{} 地级市数字治理能力 + 上级政府压力 + 同级政府压力 + 民众压力 + 上级政府明令禁止``层层加码'' + 其他控制变量

\begin{itemize}
\item
  分析单位:中国大中城市101(副省级和省会城市33 + 普通大中城市68)
\item
  变量测量

  \begin{itemize}
  \tightlist
  \item
    因变量:政策加码量

    \begin{itemize}
    \tightlist
    \item
      政策层面

      \begin{itemize}
      \tightlist
      \item
        地级市政策 - 省级政策
      \item
        地级市政策
      \end{itemize}
    \item
      执行层面

      \begin{itemize}
      \tightlist
      \item
        政务留言板的相对数量
      \item
        政务留言板的相对情绪
      \end{itemize}
    \end{itemize}
  \item
    自变量

    \begin{itemize}
    \tightlist
    \item
      核心自变量:地级市数字治理能力(2020数字政府发展指数报告)
    \item
      上级政府压力

      \begin{itemize}
      \tightlist
      \item
        上级政府开会的次数(全么?准么?)
      \item
        政府工作报告中提及的关键词数量(成果越好的提的越多?)
      \item
        该省的确诊人数
      \item
        该省的经济发展水平
      \end{itemize}
    \item
      同级政府压力

      \begin{itemize}
      \tightlist
      \item
        其他同级城市(省会和副省级城市/同省普通大中城市)的确诊数量
      \item
        其他同级城市(省会和副省级城市/同省普通大中城市)的政策严格程度
      \item
        经济差值和差值在省内的排名
      \end{itemize}
    \item
      民众压力

      \begin{itemize}
      \tightlist
      \item
        政府留言板的数量/情绪?
      \end{itemize}
    \item
      上级政府明令禁止``层层加码''
    \item
      其他控制变量

      \begin{itemize}
      \tightlist
      \item
        人口流动
      \item
        经济发展
      \item
        疫情变量
      \item
        时间控制变量
      \end{itemize}
    \end{itemize}
  \end{itemize}
\end{itemize}

\hypertarget{codebook}{%
\section{Codebook}\label{codebook}}

\hypertarget{ux5730ux65b9ux653fux7b56ux6807ux7b7e-ux7f16ux7801ux89c4ux5219}{%
\subsection{地方政策标签: 编码规则}\label{ux5730ux65b9ux653fux7b56ux6807ux7b7e-ux7f16ux7801ux89c4ux5219}}

.small{[}本节会逐个变量的进行介绍{]}

\begin{itemize}
\tightlist
\item
  1-6 测体温、健康码、健康告知书、健康问询、健康提醒、报告/报备

  \begin{itemize}
  \tightlist
  \item
    有规定 = 1
  \item
    没有规定 = 0
  \end{itemize}
\item
  7 提前报备时间

  \begin{itemize}
  \tightlist
  \item
    规定报备且有明确天数 = 直接填写提前报备的天数(单位:天)
  \item
    规定报备但无明确天数 = 1
  \item
    没有规定报备 = 0
  \end{itemize}
\item
  8 进入前核酸检测(持核酸检测阴性证明)

  \begin{itemize}
  \tightlist
  \item
    有规定 = 1
  \item
    没有规定 = 0
  \end{itemize}
\item
  9 进入前核酸检测时间

  \begin{itemize}
  \tightlist
  \item
    规定核酸检测且有明确天数 = 直接填写提前核酸检测天数(单位:天)
  \item
    规定核酸检测但无明确天数 = 7
  \item
    没有规定核酸检测 = 0
  \end{itemize}
\item
  10 进入后核酸检测

  \begin{itemize}
  \tightlist
  \item
    有规定 = 1
  \item
    没有规定 = 0
  \end{itemize}
\item
  11 进入后核酸检测次数

  \begin{itemize}
  \tightlist
  \item
    规定核酸检测且有明确次数 = 直接填写
  \item
    规定核酸检测但无明确次数 = 1
  \item
    没有规定核酸检测 = 0
  \end{itemize}
\item
  12 进入后核酸检测时间

  \begin{itemize}
  \tightlist
  \item
    规定核酸检测且有明确天数 = 直接填写(单位:天)
  \item
    规定核酸检测但无明确天数 = 1
  \item
    没有规定核酸检测 = 0
  \end{itemize}
\item
  13 血清学检测(血清抗体)

  \begin{itemize}
  \tightlist
  \item
    有规定 = 1
  \item
    没有规定 = 0
  \end{itemize}
\item
  14 居家隔离(居家健康监测)

  \begin{itemize}
  \tightlist
  \item
    有规定 = 1
  \item
    没有规定 = 0
  \end{itemize}
\item
  15 居家隔离时间

  \begin{itemize}
  \tightlist
  \item
    规定且有明确天数 = 直接填写(单位:天)
  \item
    规定但无明确天数 = 14
  \item
    无规定 = 0
  \end{itemize}
\item
  16 社区健康监测(社区健康管理)

  \begin{itemize}
  \tightlist
  \item
    有规定 = 1
  \item
    没有规定 = 0
  \end{itemize}
\item
  17 社区健康监测时间

  \begin{itemize}
  \tightlist
  \item
    规定且有明确天数 = 直接填写(单位:天)
  \item
    规定但无明确天数 = 14
  \item
    无规定 = 0
  \end{itemize}
\item
  18 集中隔离

  \begin{itemize}
  \tightlist
  \item
    有规定 = 1
  \item
    没有规定 = 0
  \end{itemize}
\item
  19 集中隔离时间

  \begin{itemize}
  \tightlist
  \item
    规定且有明确天数 = 直接填写(单位:天)
  \item
    规定但无明确天数 = 14
  \item
    无规定 = 0
  \end{itemize}
\item
  20 特殊措施
\end{itemize}

\hypertarget{dv}{%
\section{DV}\label{dv}}

\begin{itemize}
\item
  Factor Analysis?
\item
  Time?

  \begin{itemize}
  \tightlist
  \item
    测体温(1)
  \item
    健康码(1)
  \item
    健康告知书(1)
  \item
    健康问询(1)
  \item
    健康提醒(1)
  \item
    报告/报备(5)
  \item
    提前报备时间
  \item
    进入前核酸检测(60)
  \item
    进入前核酸检测时间
  \item
    进入后核酸检测(60)
  \item
    进入后核酸检测次数
  \item
    进入后核酸检测时间
  \item
    血清学检测(血清抗体)(60)
  \item
    居家隔离(360)
  \item
    居家隔离时间
  \item
    社区健康监测(720)
  \item
    社区健康监测时间
  \item
    集中隔离(1440)
  \item
    集中隔离时间
  \item
    特殊措施
  \end{itemize}
\item
  Level

  \begin{itemize}
  \tightlist
  \item
    1 = 基本通行政策(测体温、健康码、健康告知书、健康问询、健康提醒)
  \item
    2 = 健康报备
  \item
    3 = 核酸检测/抗体检测
  \item
    4 = 健康监测
  \item
    5 = 居家隔离
  \item
    6 = 集中隔离
  \end{itemize}
\end{itemize}

\hypertarget{ux4e4bux540eux7684ux7814ux7a76ux5c55ux671b}{%
\section{之后的研究展望}\label{ux4e4bux540eux7684ux7814ux7a76ux5c55ux671b}}

\begin{itemize}
\tightlist
\item
  治理能力与压力转化的关系
\item
  政治距离与压力转化的关系(测量政治优先还是人民优先)
\item
  治理能力、文化特性与疫情防控有效性的关系(究竟是能力强还是民众更守纪律)。典型事件往往可以集中放大各种矛盾。这也是为什么现在一些学者用电影来较政治学、公共管理的原因。
\end{itemize}

\end{document}
