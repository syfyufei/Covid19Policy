% Options for packages loaded elsewhere
\PassOptionsToPackage{unicode}{hyperref}
\PassOptionsToPackage{hyphens}{url}
\PassOptionsToPackage{dvipsnames,svgnames*,x11names*}{xcolor}
%
\documentclass[
  12pt,
]{ctexart}
\usepackage{amsmath,amssymb}
\usepackage{lmodern}
\usepackage{ifxetex,ifluatex}
\ifnum 0\ifxetex 1\fi\ifluatex 1\fi=0 % if pdftex
  \usepackage[T1]{fontenc}
  \usepackage[utf8]{inputenc}
  \usepackage{textcomp} % provide euro and other symbols
\else % if luatex or xetex
  \usepackage{unicode-math}
  \defaultfontfeatures{Scale=MatchLowercase}
  \defaultfontfeatures[\rmfamily]{Ligatures=TeX,Scale=1}
\fi
% Use upquote if available, for straight quotes in verbatim environments
\IfFileExists{upquote.sty}{\usepackage{upquote}}{}
\IfFileExists{microtype.sty}{% use microtype if available
  \usepackage[]{microtype}
  \UseMicrotypeSet[protrusion]{basicmath} % disable protrusion for tt fonts
}{}
\makeatletter
\@ifundefined{KOMAClassName}{% if non-KOMA class
  \IfFileExists{parskip.sty}{%
    \usepackage{parskip}
  }{% else
    \setlength{\parindent}{0pt}
    \setlength{\parskip}{6pt plus 2pt minus 1pt}}
}{% if KOMA class
  \KOMAoptions{parskip=half}}
\makeatother
\usepackage{xcolor}
\IfFileExists{xurl.sty}{\usepackage{xurl}}{} % add URL line breaks if available
\IfFileExists{bookmark.sty}{\usepackage{bookmark}}{\usepackage{hyperref}}
\hypersetup{
  pdftitle={能力不够压力来凑:数字治理能力下的压力型体制},
  pdfauthor={孙宇飞; 陈佳静},
  colorlinks=true,
  linkcolor=Maroon,
  filecolor=Maroon,
  citecolor=Blue,
  urlcolor=Blue,
  pdfcreator={LaTeX via pandoc}}
\urlstyle{same} % disable monospaced font for URLs
\usepackage[margin=1in]{geometry}
\usepackage{longtable,booktabs,array}
\usepackage{calc} % for calculating minipage widths
% Correct order of tables after \paragraph or \subparagraph
\usepackage{etoolbox}
\makeatletter
\patchcmd\longtable{\par}{\if@noskipsec\mbox{}\fi\par}{}{}
\makeatother
% Allow footnotes in longtable head/foot
\IfFileExists{footnotehyper.sty}{\usepackage{footnotehyper}}{\usepackage{footnote}}
\makesavenoteenv{longtable}
\usepackage{graphicx}
\makeatletter
\def\maxwidth{\ifdim\Gin@nat@width>\linewidth\linewidth\else\Gin@nat@width\fi}
\def\maxheight{\ifdim\Gin@nat@height>\textheight\textheight\else\Gin@nat@height\fi}
\makeatother
% Scale images if necessary, so that they will not overflow the page
% margins by default, and it is still possible to overwrite the defaults
% using explicit options in \includegraphics[width, height, ...]{}
\setkeys{Gin}{width=\maxwidth,height=\maxheight,keepaspectratio}
% Set default figure placement to htbp
\makeatletter
\def\fps@figure{htbp}
\makeatother
\setlength{\emergencystretch}{3em} % prevent overfull lines
\providecommand{\tightlist}{%
  \setlength{\itemsep}{0pt}\setlength{\parskip}{0pt}}
\setcounter{secnumdepth}{5}
\ifluatex
  \usepackage{selnolig}  % disable illegal ligatures
\fi

\title{能力不够压力来凑:数字治理能力下的压力型体制}
\usepackage{etoolbox}
\makeatletter
\providecommand{\subtitle}[1]{% add subtitle to \maketitle
  \apptocmd{\@title}{\par {\large #1 \par}}{}{}
}
\makeatother
\subtitle{------基于Covid19期间的春节返乡政策的混合方法研究}
\author{孙宇飞\footnote{清华大学政治学系博士生,联系电话:18638750921,邮箱:\href{mailto:sunyf20@mails.tsinghua.edu.cn}{\nolinkurl{sunyf20@mails.tsinghua.edu.cn}}} \and 陈佳静\footnote{美国威斯康辛大学麦迪逊分校博士生,联系电话:,邮箱:}}
\date{}

\begin{document}
\maketitle

This is a notebook recording the decisions for the project in collaboration with PhD. student Jiajing Chen to study the Covid-19 mandatory policy during the Spring Festival of 2021 in prefecture-level cities in China.

\hypertarget{ux6587ux7ae0ux5206ux6790ux903bux8f91}{%
\section{文章分析逻辑}\label{ux6587ux7ae0ux5206ux6790ux903bux8f91}}

政策加码过程:中央 - 省 - 地级市 - 基层

政策加码主体:下一级政府

中央 - \textbf{省}:公开政策比较

省 - \textbf{地级市}-政策的层面:公开政策比较

省 - \textbf{地级市}-执行的层面:公开政策与政务留言板的数值差异比较

\hypertarget{ux5206ux6790ux6a21ux578b}{%
\section{分析模型}\label{ux5206ux6790ux6a21ux578b}}

政策加码量(地级市) \textasciitilde{} 地级市数字治理能力 + 上级政府压力 + 同级政府压力 + 民众压力 + 其他控制变量

\begin{itemize}
\item
  分析单位:中国大中城市101(副省级和省会城市33 + 普通大中城市68)
\item
  变量测量

  \begin{itemize}
  \tightlist
  \item
    因变量:政策加码量

    \begin{itemize}
    \tightlist
    \item
      政策层面

      \begin{itemize}
      \tightlist
      \item
        地级市政策 - 省级政策
      \item
        地级市政策
      \end{itemize}
    \item
      执行层面

      \begin{itemize}
      \tightlist
      \item
        政务留言板的相对数量
      \item
        政务留言板的相对情绪
      \end{itemize}
    \end{itemize}
  \item
    自变量

    \begin{itemize}
    \tightlist
    \item
      核心自变量:地级市数字治理能力(2020数字政府发展指数报告)
    \item
      上级政府压力

      \begin{itemize}
      \tightlist
      \item
        上级政府开会的次数
      \item
        政府工作报告中提及的关键词数量
      \item
        该省的确诊人数
      \end{itemize}
    \item
      同级政府压力

      \begin{itemize}
      \tightlist
      \item
        其他同级城市(省会和副省级城市/同省普通大中城市)的确诊数量
      \item
        其他同级城市(省会和副省级城市/同省普通大中城市)的政策严格程度
      \end{itemize}
    \item
      民众压力

      \begin{itemize}
      \tightlist
      \item
        政府留言板的数量/情绪?
      \end{itemize}
    \item
      其他控制变量

      \begin{itemize}
      \tightlist
      \item
        人口流动
      \item
        经济发展
      \item
        疫情变量
      \end{itemize}
    \end{itemize}
  \end{itemize}
\end{itemize}

\end{document}
